\chapter{Preliminaries}\label{preliminaries}
In this chapter, we present the theoretical background and applications of knowledge graphs and graph representation learning. These concepts will be useful to understand the research presented in the following chapters.

\section{Knowledge graphs}\label{prel:kgs}
We begin by defining what knowledge graphs are in Section~\ref{kgs:definition}, followed by the applications of these graphs in biomedicine in Section~\ref{kgs:biomed}.

\subsection{Definition}\label{kgs:definition}
The concept of knowledge graphs has been a subject of research since 2012, especially in the Semantic Web community, but despite their widespread use, a unified definition remains elusive. Various definitions from literature demonstrate significant variation in scope and focus. Below we present a selection of definitions to illustrate this diversity:
\begin{description}
    \item[Structure and Composition] Paulheim~\cite{Paulheim2016KnowledgeGR} (2015) proposed a minimal set of characteristics that distinguish KGs as graph-based knowledge representations:``A knowledge graph (i) mainly describes real world entities and their interrelations, organized in a graph, (ii) defines possible classes and relations of entities in a schema, (iii) allows for potentially interrelating arbitrary entities with each other and (iv) covers various topical domains''. Similarly, the Journal of Web Semantics describe KGs as ``graphs of entities, their types, and relationships between entities''~\cite{Kroetsch2016knowledge}
    \item[Domain-Specific Networks] The Semantic Web Company highlights the flexibility of KGs to represent domain-specific networks, incorporating not just concepts but also instances like documents and datasets.~\cite{Blumauer2014knowledge}
    \item[Formal Definition] As proposed by Färber et al. in ~\cite{Farber2016linked} knowledge graph can be defined as an RDF graph. An RDF graph consists of a set of triples $(s, p, o)$ is an ordered set of the following RDF terms: 
    \begin{itemize}
        \item a subject $s\in U \cup B$
        \item an object $o\in U\cup B\cup L$
        \item a predicate $p\in U$
    \end{itemize}
    An RDF term is either a URI $u\in U$, a blank node $b\in B$, or a literal $l\in L$.”
    \item[Dynamic Extraction] Pujara et al.~\cite{Pujara2013KnowledgeGF} focus on the automatic extraction of knowledge from the web in the form of facts that are interrelated.
\end{description}

This diversity in definitions hinders precise discussions and comparative research, emphasizing the need for a common, comprehensive definition to advance the field. To address this gap, the authors in~\cite{Ehrlinger2016TowardsAD} propose a new definition of a knowledge graph: 
\begin{center}
    \begin{quote}
        \textit{A knowledge graph acquires and integrates information into an ontology and applies a reasoner to derive new knowledge.}
    \end{quote}
\end{center}This definition seeks to unify the understanding of KGs by emphasizing three critical features:
\begin{description}
    \item[Information integration] KGs collect, extract, and integrate information from multiple external sources, making them dynamic and scalable.
    \item[Reasoning capabilities] The inclusion of a reasoning engine allows KGs to infer new knowledge from existing data, which elevates their functionality beyond a static repository.
    \item[Ontological foundation] KGs are built on an ontological foundation, which provides a formal structure for representing knowledge and relationships between entities.
\end{description}

The authors stress that a KG is more complex than a traditional knowledge base or ontology, as it combines these features to support automated reasoning and dynamic knowledge integration. The size of the KG is considered less critical compared to its functional capabilities.

\subsection{Applications in biomedicine}\label{kgs:biomed}

\section{Graph Representation Learning}